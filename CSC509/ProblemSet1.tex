%%This is a very basic article template.
%%There is just one section and two subsections.
\documentclass{article}
\usepackage[margin=0.5in]{geometry}
\usepackage{listings}
\usepackage{color}

\definecolor{dkgreen}{rgb}{0,0.6,0}
\definecolor{gray}{rgb}{0.5,0.5,0.5}
\definecolor{mauve}{rgb}{0.58,0,0.82}

\lstset{frame=tb,
  language=C,
  aboveskip=3mm,
  belowskip=3mm,
  showstringspaces=false,
  columns=flexible,
  basicstyle={\small\ttfamily},
  numbers=none,
  numberstyle=\tiny\color{gray},
  keywordstyle=\color{blue},
  commentstyle=\color{dkgreen},
  stringstyle=\color{mauve},
  breaklines=true,
  breakatwhitespace=true,
  tabsize=3
}

\begin{document}


\title{Problem Set 1}
\author{Greg Timmons (gbtimmon) }
\date{January 19, 2015}
\maketitle
\section{Problem 1}
If $x$ is the number of processor and y is the desired speed up then the following equation applies. 
\[ 16 + (\frac{84}{x}) = (\frac{100}{y}) \rightarrow x = 84(\frac{100}{y} - 16)^{-1}\]
\begin{description}
  \item[(a)] 
  Setting $y$ to 2 gives $x = 2.64$, or 3 processors. 
  \item[(b)]
  Setting $y$ to 3 gives $x = 4.84$, or 5 processors.  
  \item[(c)]
  Setting $y$ to 5 gives $x = 21.00$, or 21 processors. 
  \item[(d)]
  Setting $y$ to 7 results in a negative x, and it would not be possible to reach this speed up. This is because a speed up of 7 requires a total exectution
  time of 14.2 seconds, however there is a lower limit of 16 due to the serial portion of the program. 
\end{description}
\section{Problem 2}
\begin{lstlisting}
for( i = 1;  i <= 1024; i++ ) { 
    sum[i] = 0; 
    for ( j = 1;  j <= i; j++ ) { 
        sum[i] = sum[i] +i;
    }
}
\end{lstlisting}
\begin{description}
	\item[(a)] The total excution time will be 
	\[\sum\limits_{i=1}^{n}{2 + 2i}\]
	\[= \sum\limits_{i=0}^{n}{2} + 2S\sum\limits_{i=0}^{n}{i}\]
	\[= 2n + 2\frac{(n^2 + n)}{2}\]
	\[= 1,051,648 \quad cycles\]
	\item[(b)] Processor k will process steps $32i-31$ to $32i$. The excution time $E(k)$ can be computer as follows.
	\[E(k) = \sum\limits_{i=(32k - 31)}^{32k}{2 + 2i}\]
	\[= \sum\limits_{i=1}^{32k}{2 + 2i} -\sum\limits_{i=1}^{32(k-1)}{2 + 2i}\]
	\[= (\sum\limits_{i=1}^{32k}{2} - \sum\limits_{i=1}^{32(k-1)}{2}  ) + (2\sum\limits_{i=1}^{32i}{i} - 2\sum\limits_{i=1}^{32(k-1)}{i})\]
	\[64 + (32k)(32k + 1 ) - (32(k-1))(32(k-1) + 1)\]
	\[2048k - 928\]
	As this equation is strictly increasing, we know that k=32 will be our slowest process and it pluging 32 in for k we get a runtime of 
	\item[(c)] A semi modified round robin assignent pattern can result in optimal scheduling for this work. Psuedocode follows : 
	\begin{lstlisting}
	
	processor = [0] * 32 
	for ( i in range(1024) ):
	    index = i % 64
	    if (index >= 32) : 
	        index = 63 - index 
	    processor[index] = processor[index] + 2i + 2       
	\end{lstlisting}
	
	In this assignment pattern all processors will recieve 32,800 units of work, which is the minimum exctution time. 
	\\The speed up is $\frac{1051648}{32800}$ or 32.8 times faster.  
	
\end{description}
\section{Problem 3}


\end{document}
