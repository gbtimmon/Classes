%%This is a very basic article template.
%%There is just one section and two subsections.
\documentclass{article}
\usepackage{amsmath}
\usepackage{pgf}
\usepackage{lastpage}
\usepackage{amssymb}
\usepackage{tikz}
\usepackage[margin=0.75in]{geometry}
\usetikzlibrary{arrows,matrix,positioning}
\usepackage{listings}             % Include the  listings-package
\usepackage[utf8]{inputenc}
\usepackage[english]{babel}
\usepackage{fancyhdr}
\lstset{language=Matlab} 
\pagestyle{fancy}
\fancyhf{}
\fancyhead[LE,RO]{CTMC - Greg Timmons}
\fancyhead[RE,LO]{CSC 579 - Perf Modeling}
\fancypagestyle{plain}{\fancyfoot[LE,RO]{\thepage\backslash\pageref{LastPage}}} 
\fancyfoot[LE,RO]{\thepage\backslash\pageref{LastPage}}  
\usetikzlibrary{arrows,automata}
\usepackage[latin1]{inputenc}
\usepackage{pdfpages}
\begin{document}
	\title{Class Notes -- CTMC}
	\date{2/28/2015}
	\author{Gregory B Timmons, gbtimmon}
	\maketitle
\section*{\underline{CTMC Basics}} \\
 Continuous Time Markov chain, is a markov chain is a chain
where a state transtion can happen at infinitesimally small time periods -- at any
point in continuous time. We measure tranistion in the expected rate of
transitions.
\\
\\This markov chain is still \underline{memoryless}, in that neither previous
states or time spent in the current state relavent to the computation of
probability of transition to furture states
\\
\\Non-homogenous CTMC still depend on time in the chain.
\underline{$p_{ij}(s,t)$} is the probability having transitioned from state 'i' to state 'j' by time 't' when you
where in state i at time 's'.
\[P_{ij}(s,t) = \mbox{Prob}\{X(t) = j | X(s) = i\} \]
\\Similar to a DTMC, a homogenous chain does not depend on how long you have
been in the chain as transtion probabilities are not dependent on time. 's' and
;t' from above can simply be expressed as a amount of time passed, $\tau$.
\[P_{ij}(\tau) = \mbox{Prob}\{X(s + \tau) = j | X(s) = i\}, \quad \forall s \geq
0\]
\\
Tranistions probability adjust to the time period being examined. They always
sum to one regarless to the size of $\tau$.
\[\sum\limits_{\mbox{all j}}{p_{ij}(\tau) = 1, \quad \forall \tau \geq 0}\]
\section*{\underline{Transition Rates vs Transition Probabilities}}
Where a DTMC is represented in transtion probabilities, CTMC are dependent on
the amount of time being examined $\tau$. A DTMC can be seen as a CTMC where
$\tau$ is restricted to be integers or unit times. But for the CTMC we need to
support $\tau$ in the real number space so tranistion probabilities become
unatural to express -- rather we more naturally express these markov chains in
expected rates of transition -- That is 5 per hour on average, which are not
dependent on the observation time. 
\\
\\
There are intersting observations about the size of $\tau$. As $\tau
\rightarrow 0$, $p_{ii}$ goes to 1, and the sum of all other probabilities goes
to 0. Assuming i is not an abosrbing state, as $tau$ goes to infinity, then
$P_{ii}$ goes to 0 and the sum of all other probabilities in row goes to 1. 
\\
\\We define $q_{ij}$ to be the transition rate from i to j. In a non-homogenous
sense this quantity can depend on 't' but it does not depend on the length of
time $\tau$ being observed. 
\\
\\Q is a speed of change of the probability of transition as time moves. 
\\
\\for $q_ii$ the transition rate is negative, to imply as time increases the
chance to stay decreases. 
\\When 'i' is an absorbing state $q_{ii}$ is 0. since the derivative of 0

\end{document}