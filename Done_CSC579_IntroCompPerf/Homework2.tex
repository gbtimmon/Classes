%%This is a very basic article template.
%%There is just one section and two subsections.
\documentclass{article}
\usepackage{amsmath}
\usepackage{pgf}
\usepackage{lastpage}
\usepackage{amssymb}
\usepackage{tikz}
\usepackage[margin=0.75in]{geometry}
\usetikzlibrary{arrows,matrix,positioning}
\usepackage{listings}             % Include the  listings-package
\usepackage[utf8]{inputenc}
\usepackage[english]{babel}
\usepackage{fancyhdr}
 
\lstset{language=Matlab} 
\pagestyle{fancy}
\fancyhf{}
\fancyhead[LE,RO]{Homework 1 - Greg Timmons}
\fancyhead[RE,LO]{CSC 579 - Perf Modeling}
\fancypagestyle{plain}{
\fancyfoot[LE,RO]{\thepage\backslash\pageref{LastPage}}  
} 
\fancyfoot[LE,RO]{\thepage\backslash\pageref{LastPage}}  
\usetikzlibrary{arrows,automata}
\usepackage[latin1]{inputenc}
\usepackage{pdfpages}



\begin{document}
\includepdf[pages={1}]{cover.pdf}
\title{Homework 2}
\date{Febuary 26, 2015}
\author{Gregory B Timmons, gbtimmon}
\maketitle
\section*{Question 1}
\[ r_{ij} = \frac{p_{ji}\pi_j}{\pi_i}, \quad R = \mbox{diag}\{\pi\}^{-1} P^{T}
\mbox{diag}\{\pi\}.\quad \mbox{ Where }\pi \mbox{ is a stationary distribution
of P.}\]
\subsection*{\underline{$P_1$}}
\[ \pi_1 P_1 =  \left( 1/3, 1/3, 1/3 \right) \left( \begin{array}{ccc}
	0&1&0\\
	0&0&1\\
    1&0&0\\
\end{array}\right)  = \left( 1/3, 1/3, 1/3 \right) = \pi_1\]

\[
\left( \begin{array}{ccc}
	3&0&0\\
	0&3&0\\
    0&0&3\\
\end{array}\right)
\left( \begin{array}{ccc}
	0&1&0\\
	0&0&1\\
    1&0&0\\
\end{array}\right)^{T}
\left( \begin{array}{ccc}
	1/3	&0&0\\
	0&1/3&0\\
    0&0&1/3
    \\
\end{array}\right) = 
\left( \begin{array}{ccc}
	0&0&1\\
	1&0&0\\
    0&1&0\\
\end{array}\right)
\]

\subsection*{\underline{$P_2$}}
\[ \pi_2 P_2 =  \left( 1/3, 1/3, 1/3 \right) \left( \begin{array}{ccc}
	.4&.2&.4\\
	.1&.3&.6\\
    .5&.5& 0\\
\end{array}\right)  = \left( 1/3, 1/3, 1/3 \right) = \pi_2\]

\[
\left( \begin{array}{ccc}
	3&0&0\\
	0&3&0\\
    0&0&3
\end{array}\right)
\left( \begin{array}{ccc}
	.4&.2&.4\\
	.1&.3&.6\\
    .5&.5& 0
\end{array}\right)^{T} \left( \begin{array}{ccc}
	1/3	&0&0\\
	0&1/3&0\\
    0&0&1/3
\end{array}\right) = 
\left( \begin{array}{ccc}
	.4&.1&.5\\
	.2&.3&.5\\
    .4&.6&0\\
\end{array}\right)
\]
\subsection*{\underline{$P_3$}}
\[ \pi_3 P_3 =  \left( 1/4, 5/12, 1/4, 5/12 \right) \left( \begin{array}{cccc}
	0&1&0&0\\
	0&0&.6&.4\\
    0&0&0&1\\
    .6&.4&0&0
\end{array}\right)  = \left( 1/4, 5/12, 1/4, 5/12 \right) = \pi_3\]

\[
\left( \begin{array}{cccC}
	4&    0& 0&    0\\
	0& 12/5& 0&    0\\
    0&    0& 4&    0\\
    0&    0& 0& 12/5
\end{array}\right)
\left(\begin{array}{cccc}
	  0&  1&  0&  0\\
      0&  0& .6& .4\\
      0&  0&  0&  1\\
     .6& .4&  0&  0
\end{array}\right) ^{T}
\left( \begin{array}{cccc}
	1/4 & 0    &   0 &    0 \\
	0   & 5/12 &   0 &    0 \\
    0   & 0    & 1/4 &    0 \\
    0   & 0    &   0 & 5/12 
\end{array}\right) = 
\left( \begin{array}{cccc}l
	 0 &  0 &  0 &  1 \\
	.6 &  0 &  0 & .4 \\
     0 &  1 &  0 &  0 \\
     0 & .4 & .6 &  0
\end{array}\right)
\]
\section*{Question 2}
\subsection*{\underline{$P_1$}}
This markov chain is not reversible because it does not have a symmetirc
structure, that is $ \exists $ $i, j $ s.t. $ p_{ij} > 0 $ and $p_{ji} = 0$
\subsection*{\underline{$P_2$}}
\(P_2 \mbox{ has a stationary distribution } \pi = \left(\begin{array}{cccc}
 0.13 & 0.30 & 0.33 & 0.33 \\
\end{array}\right)\)
\\
\\
\[\mbox{diag}\{\pi\}^{-1}
\left(\begin{array}{cccc}
 0.00 & 0.75 & 0.25 & 0.00 \\
 0.33 & 0.00 & 0.44 & 0.22 \\
 0.10 & 0.40 & 0.00 & 0.50 \\
 0.00 & 0.29 & 0.71 & 0.00 \\
\end{array}\right)^{T}
\mbox{diag}\{\pi\} = 
\left(\begin{array}{cccc}
 0.00 & 0.75 & 0.25 & 0.00 \\
 0.33 & 0.00 & 0.44 & 0.22 \\
 0.10 & 0.40 & 0.00 & 0.50 \\
 0.00 & 0.29 & 0.71 & 0.00 \\
\end{array}\right)\]
\\
\\
From the above derivation, we can see the $P = R$, which implies the detail
equations are satisfied and the markov chain is reversible. 
\subsection*{\underline{$P_3$}}
\(P_3 \mbox{ has a stationary distribution } \pi = \left(\begin{array}{cccccc}
 0.12 & 0.14 & 0.08 & 0.27 & 0.27 & 0.11 \\
\end{array}\right)

\\
\\
\[\mbox{diag}\{\pi\}^{-1}
\left(\begin{array}{cccccc}
 0.10 & 0 & 0 & 0.90 & 0 & 0 \\
 0 & 0 & 0 & 0 & 1 & 0 \\
 0 & 0 & 0 & 0 & 1 & 0 \\
 0.40 & 0 & 0 & 0 & 0.20 & 0.40 \\
 0 & 0.50 & 0.30 & 0.20 & 0 & 0 \\
 0 & 0 & 0 & 1 & 0 & 0 \\
\end{array}\right)^{T}
\mbox{diag}\{\pi\} = 
\left(\begin{array}{cccccc}
 0.10 & 0 & 0 & 0.90 & 0 & 0 \\
 0 & 0 & 0 & 0 & 1 & 0 \\
 0 & 0 & 0 & 0 & 1 & 0 \\
 0.40 & 0 & 0 & 0 & 0.20 & 0.40 \\
 0 & 0.50 & 0.30 & 0.20 & 0 & 0 \\
 0 & 0 & 0 & 1 & 0 & 0 \\
\end{array}\right)
\]
From the above derivation, we can see the $P = R$, which implies the detail
equations are satisfied and the markov chain is reversible. 

\subsection*{\underline{$P_4$}}
\(P_3 \mbox{ has a stationary distribution } \pi = \left(\begin{array}{ccc}
 0.35 & 0.21 & 0.44 \\
\end{array}\right)

\\
\\
\[\mbox{diag}\{\pi\}^{-1}
\left(\begin{array}{ccc}
 0.50 & 0.10 & 0.40 \\
 0.20 & 0.40 & 0.40 \\
 0.30 & 0.20 & 0.50 \\
\end{array}\right)^{T}
\mbox{diag}\{\pi\} = 
\left(\begin{array}{ccc}
 0.50 & 0.12 & 0.38 \\
 0.17 & 0.40 & 0.43 \\
 0.31 & 0.19 & 0.50 \\
\end{array}\right)


\]
From the above derivation, we can see the $P \neq R$, which implies the detail
balance equations are not upheld and the markov chain is not reversible. 

\section*{Question 3}
\subsection*{\bold{a.)}}
By representing this as a CTMC, we assume no to units can break at the same
time, nor can a unit break at the same time that one is repaired. 
\subsection*{\bold{b.)}} State 1 indicates that all of the peices are working as expected. 
\\States 2,3,4 indicate the number of damaged computers, 1 2 or 3 respectively. 
\\State 5 indicates a damaged voter. 
\\If a computer is damaged we move into a higher state aprroaching state 4. Once
in state 4, all computers are broken and no more can be broken. At any point in
states 1 through 4, the vote may break. If the voter breaks we go to state 5 ,
waiting for repair. Since we are garunteed that all computers are repaired with
the voter when the voter is repaired, then state 5 will return to state one on
repair. 
\\
\\All states should be recurrent and the markov chain would be irreducible. 
 \subsection*{\bold{c.)}} 
 \begin{tikzpicture}[->,>=stealth', shorten >=5pt,auto,node distance=3cm, thick]
 \tikzstyle{every state}=[rectangle,rounded corners,draw,align=center, minimum size=1.25cm]

  \node[state]         (0)                    {$S_0$};
  \node[state]         (1) [right of=0]       {$S_1$};
  \node[state]         (2) [right of=1]       {$S_2$};
  \node[state]         (3) [right of=2]       {$S_3$};
  \node[state]         (4) [below of=1]       {$S_4$};

  \path (0) edge   [bend left =15]    node {$\lambda$} (1)
        (1) edge   [bend left =15]    node {$\mu$} (0)
        (1) edge   [bend left =15]    node {$\lambda$} (2)
        (2) edge   [bend left =15]    node {$\mu$} (1)
        (2) edge   [bend left =15]    node {$\lambda$} (3)
        (3) edge   [bend left =15]    node {$\mu$} (2)
        (0) edge   [bend right=45]    node {$\epsilon$} (4)
        (4) edge   [bend right=10]    node {$\upsilon$} (0)
        (1) edge   [bend left =15]    node {$\epsilon$} (4)
        (2) edge   [bend left =15]    node {$\epsilon$} (4)
        (3) edge   [bend left =15]    node {$\epsilon$} (4)
\end{tikzpicture}
\subsection*{\bold{d.)}}        
\[ Q = \left(\begin{array}{ccccc}
 $-(\lambda + \epsilon)$ & $\lambda$ & 0 & 0 & $\epsilon$\\
 $\mu$ & $-(\mu + \lambda + \epsilon)$ & $\lambda$ & 0 & $\epsilon$\\
 0 & $\mu$ & $-(\mu + \lambda + \epsilon)$ & $\lambda$ & $\epsilon$\\
 0 & 0 & $\mu$ & $-(\mu + \epsilon)$ & $\epsilon$\\
 $\upsilon$ & 0 & 0 & 0 & -$\upsilon$\\
\end{array}\right)\]
\section*{Question 4} 
The CTMC below describes this process. Once every 9 hours we expect a machine to
break. This can be represented by the below CTMC with states $S_i$ where $i$
indicates the number of damaged machines and 

\begin{tikzpicture}[->,>=stealth', shorten >=5pt,auto,node distance=3cm, thick]
\tikzstyle{every state}=[rectangle,rounded corners,draw,align=center,
minimum size=1.25cm]

  \node[state]         (0)                    {$S_0$};
  \node[state]         (1) [right of=0]       {$S_1$};
  \node[state]         (2) [right of=1]       {$S_2$};
  \node[state]         (3) [right of=2]       {$S_3$};

  \path (0) edge   [bend left =15]    node {$\frac{1}{9}$} (1)
        (1) edge   [bend left =15]    node {$\frac{1}{2}$} (0)
        (1) edge   [bend left =15]    node {$\frac{1}{9}$} (2)
        (2) edge   [bend left =15]    node {$\frac{1}{2}$} (1)
        (2) edge   [bend left =15]    node {$\frac{1}{9}$} (3)
        (3) edge   [bend left =15]    node {$\frac{1}{2}$} (2)
\end{tikzpicture}

This has a transition rate matix of 
\[ Q = \left(\begin{array}{cccc}
 -1/9 & 1/9 & 0 & 0 \\
 1/2 & -11/18 & 1/9 & 0 \\
 0 & 1/2 & -11/18 & 1/9 \\
 0 & 0 & 1/2 & -1/2 \\
\end{array}\right)\]

I will increase the rate by a linear scalar (18) to simplify computations. This
should not effect steady state computations. 

\[ Q = \left(\begin{array}{cccc}
 -2 & 2 & 0 & 0 \\
 9 & -11 & 2 & 0 \\
 0 & 9 & -11 & 2 \\
 0 & 0 & 9 & -9 \\
\end{array}\right)\]
To find a steady state for this Transition matrix we find a $\pi$ such that
$\piQ = 0$.
\[
\left(\begin{array}{cccc}
\pi_1 &\pi_2 & \pi_3 & \pi_4 
\end{array}\right)
\left(\begin{array}{cccc}
 -2 & 2 & 0 & 0 \\
 9 & -11 & 2 & 0 \\
 0 & 9 & -11 & 2 \\
 0 & 0 & 9 & -9 \\
\end{array}\right) = \left(\begin{array}{cccc}
0 & 0 & 0 & 0
\end{array}\right)\]	
This give rise to the following system of equations  :
\[ -2\pi_1 + 9\pi_2 = 0 \]
\[ 2\pi_1 - 11\pi_2 + 9\pi_3 = 0 \]
\[ 2\pi_2 - 11\pi_3 + 9\pi_4 = 0 \]
\[ 2\pi_3 - 9\pi_4 = 0 \]
Solving 
\[\mbox{ let } \boxed{\pi_1 = 1}\]
\[-2 + 9\pi_2 = 0 \Rightarrow \boxed{\pi_2 = 2/9} \]
\[2 - 22/9 + 9\pi_3 = 0 \Rightarrow \boxed{\pi_3 = 4/81} \]
\[ 8/81 - 9\pi_4 = 0 \Rightarrow \boxed{\pi_4 = 8/726} \]
Leading to
\[ \pi = \left(\begin{array}{cccc}
1 & 2/9 & 4/81 & 8/726
\end{array}\right)\]
Which normalized is 
\[\left(\begin{array}{cccc}
 0.780 & 0.173 & 0.039 & 0.009 \\
\end{array}\right)\]
The expeceted number of broken machines is computeed below
\[\left(\begin{array}{cccc}
 0.780 & 0.173 & 0.039 & 0.009 \\
\end{array}\right) * \[\left(\begin{array}{cccc}
0&1&2&3\\
\end{array}\right)^{T} = 0.276\]
 \section*{Question 5}   
Let $pi = \left(\begin{array}{cccccc}0&0&0&1&0&0\end{array}\right)$
\\$P(t)$ is computed by $\mbox{expm}\left(Q\right)^t$
\\The states of interest are highlighted $(1, 4, 6h)$
\[pi * P(0.01) = 
\left(\begin{array}{cccccc}
 \underline{0} & 0.0011 & 0.0557 & \underline{0.9156} & 0.0270 &
 \underline{0.0006}
\end{array}\right)
\] 
\[pi * P(0.1) = 
\left(\begin{array}{cccccc}
 \underline{0.0052} & 0.0683 & 0.3025 & \underline{0.4801} & 0.1122 &
 \underline{0.0316} \\
\end{array}\right)
\] 
\[pi * P(0.5) = 
\left(\begin{array}{cccccc}
 \underline{0.1777} & 0.2889 & 0.2286 & \underline{0.1172} & 0.0382 &
 \underline{0.1494} \\
\end{array}\right)
\] 
\[pi * P(1.0) = 
\left(\begin{array}{cccccc}
 \underline{0.4379} & 0.2112 & 0.1029 & \underline{0.0420} & 0.0124 &
 \underline{0.1935} \\
\end{array}\right)
\] 
\[pi * P(2.0) = 
\left(\begin{array}{cccccc}
 \underline{0.6842} & 0.0605 & 0.0252 & \underline{0.0094} & 0.0027 &
 \underline{0.2180} \\
\end{array}\right)
\] 
\[pi * P(5.0) = 
\left(\begin{array}{cccccc}
 \underline{0.7726} & 0.0011 & 0.0004 & \underline{0.0002} & 0.0000 &
 \underline{0.2257} \\
\end{array}\right)
\] 

\section*{Question 6}
\subsection*{a.)}
The embeded markov chain for this chain is very simple : 
\[EMC(Q) = h
\left(\begin{array}{cccccc}
 0 & 1 & 0 & 0 & 0 & 0 \\
 0 & 0 & 1 & 0 & 0 & 0 \\
 0 & 0 & 0 & 1 & 0 & 0 \\
 0 & 0 & 0 & 0 & 1 & 0 \\
 0 & 0 & 0 & 0 & 0 & 1 \\
 1 & 0 & 0 & 0 & 0 & 0 \\
\end{array}\right)
\]
Its trivial to see from this that an even distirbution in the markov chian will
be will be a stationary distribution.
\[\pi = \left(\begin{array}{cccccc}
 0.167 & 0.167 & 0.167 & 0.167 & 0.167 & 0.167 \\
\end{array}\right)\]
Divide by the origianl dagonal rates : 

\[\pi = \left(\begin{array}{cccccc}
 0.167h & 0.167 & 0.167 & 0.167 & 0.167 & 0.167 \\
\end{array}\right)\]
\subsection*{b.)}

\[
\left(\begin{array}{cccccc}
\pi_1 &\pi_2 & \pi_3 & \pi_4 & \pi_5 & \pi_6
\end{array}\right)
\left(\begin{array}{cccccc}
 -1 & 1 & 0 & 0 & 0 & 0 \\
 0 & -2 & 2 & 0 & 0 & 0 \\
 0 & 0 & -3 & 3 & 0 & 0 \\
 0 & 0 & 0 & -4 & 4 & 0 \\
 0 & 0 & 0 & 0 & -5 & 5 \\
 6 & 0 & 0 & 0 & 0 & -6 \\
\end{array}\right)\]
Leads to the system of equations :
\[ -\pi_1 + \pi_6 = 0 \]
\[  \pi_1 - 2\pi_2 = 0 \]
\[ 2\pi_2 - 3\pi_3 = 0 \]
\[ 3\pi_3 - 4\pi_4 = 0 \]
\[ 4\pi_4 - 5\pi_5 = 0 \]
\[ 5\pi_5 - 6\pi_6 = 0 \]
Solving : 
\[ \mbox{let } \boxed{\pi_1 = 1 }\]
\[  1 - 2\pi_2 = 0 \Rightarrow \boxed{\pi_2 = 1/2} \]
\[  1 - 3\pi_3 = 0 \Rightarrow \boxed{\pi_3 = 1/3} \]
\[  1 - 4\pi_4 = 0 \Rightarrow \boxed{\pi_4 = 1/4} \]
\[  1 - 5\pi_5 = 0 \Rightarrow \boxed{\pi_5 = 1/5} \]
\[  1 - 5\pi_5 = 0 \Rightarrow \boxed{\pi_5 = 1/5} \]
\[  1 - 6\pi_6 = 0 \Rightarrow \boxed{\pi_5 = 1/6} \]
Giving the unormalized vector:
\[\pi = \left(\begin{array}{cccccc}
 1& 1/2 & 1/3 & 1/4 & 1/5 & 1/6
\end{array}\right)\]
And the normalize vector: 
\[\pi = \left(\begin{array}{cccccc}
 0.408 & 0.204 & 0.136 & 0.102 & 0.082 & 0.068 \\
\end{array}\right)\]
\end{array}
\end{document}