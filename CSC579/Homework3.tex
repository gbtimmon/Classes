%%This is a very basic article template.
%%There is just one section and two subsections.
\documentclass{article}
\usepackage{amsmath}
\usepackage{pgf}
\usepackage{lastpage}
\usepackage{amssymb}
\usepackage{tikz}
\usepackage[margin=0.75in]{geometry}
\usetikzlibrary{arrows,matrix,positioning}
\usepackage{listings}             % Include the  listings-package
\usepackage[utf8]{inputenc}
\usepackage[english]{babel}
\usepackage{fancyhdr}
 
\lstset{language=Matlab} 
\pagestyle{fancy}
\fancyhf{}
\fancyhead[LE,RO]{Homework 3 - Greg Timmons}
\fancyhead[RE,LO]{CSC 579 - Perf Modeling}
\fancypagestyle{plain}{
\fancyfoot[LE,RO]{\thepage\backslash\pageref{LastPage}}  
} 
\fancyfoot[LE,RO]{\thepage\backslash\pageref{LastPage}}  
\usetikzlibrary{arrows,automata}
\usepackage[latin1]{inputenc}
\usepackage{pdfpages}



\begin{document}
\includepdf[pages={1}]{cover.pdf}
\title{Homework 3}
\date{March 22, 2015}
\author{Gregory B Timmons, gbtimmon}
\maketitle
\section*{Question 1}
\[ P = 
\left(\begin{array}{ccccc}
0.2   & 0.0   & 0.005 & 0.795 & 0.0   \\
0.0   & 0.0   & 0.998 & 0.002 & 0.0   \\
0.002 & 0.0   & 0.0   & 0.0   & 0.998 \\
0.8   & 0.001 & 0.0   & 0.198 & 0.001 \\
0.0   & 0.998 & 0.0   & 0.002 & 0.0   
\end{array}\right)
\]
\\	
\[ P_{HB} = \]\[
\begin{array}{ccccccccccccc}
0.2 &0.005 &0.795 &0.998 &0.002 &0.002 &0.998 &0.8 &0.001 &0.198 &0.001 &0.998 &0.002 \\
1 &3 &4 &3 &4 &1 &5 &1 &2 &4 &5 &2 &4 \\
1 &4 &6 &8 &12 &14 
\end{array}
\]

\[ Q_{HB} = (P^{T} - I)^{T}_{HB} = \]\[
\begin{array}{cccccccccccccccc}
-0.8 &0.005 &0.795 &-1 &0.998 &0.002 &0.002 &-1 &0.998 &0.8 &0.001 &-0.802
&0.001 &0.998 &0.002 &-1 \\
1 &3 &4 &2 &3 &4 &1 &3 &5 &1 &2 &4 &5 &2 &4 &5 \\
1 &4 &7 &10 &14 &17 
\end{array}\]
\section*{Question 2}
The Gauss-Siedel was computed on the Harwell Boeing matrix using the following
matlab code 
\begin{lstlisting}[frame=single] 
 function pi = HW3_gaussSeidel(pi, V, J, I)
    for s = 2 : size(I, 2)  
        sum = 0;
        for i = I(s-1):I(s)-1
            sum = sum + pi(J(i))*V(i);
        end  
        pi(s-1) = sum;
    end
end 
\end{lstlisting}
Using the values \\
$ V = \left(\begin{array}{ccccccc}
	0.4 & 0.6 & 0.4 & 0.4 & 0.2 & 0.2 & 0.8 \\
\end{array}\right)$\\
$ J    = \left(\begin{array}{ccccccc}
	3 & 1 & 2 & 1 & 3 & 3 & 2 \\
\end{array}\right)$\\
$ I   = \left(\begin{array}{cccc}
	1 & 3 & 6 & 8 \\
\end{array}\right)$\\
\\
with \\
$\pi^{(0)} = \left(\begin{array}{ccc}
	0 & 1 & 0
\end{array}\right)$ \\ 
\\
I computed the following values :\\
$\pi^{(1)} = \left(\begin{array}{ccc}
	0  &  0.5556  &  0.4444
\end{array}\right)$ \\ 
$\pi^{(2)} = \left(\begin{array}{ccc}
0.3135   & 0.3407  &  0.3459
\end{array}\right)$ \\ 

\section*{Question 3}
Interarrival time $1/\lambda = .25$ -- mean arrival rate
$\lambda = 4$ \subsection*{(a)}
\[ p_0(t) = e^{-\lambda t}\]
\[ p_{0}(.5) = e^{-(2)} = 0.135 \]
\subsection*{(b)}
At a rate of 4 per hours, we expect to reach ten customers at 2.5 hours. 
\subsection*{(c)}
\[\mbox{Prob}\{N(t) = n \} = p_n(t) = e^{-\lambda t}\frac{(\lambda
t)^{n}}^{n!}\]
\[\mbox{Prob}\{N(.5) > 5 \} = 1 - \mbox{Prob}\{N(.5) \leq 5 \} = 1 -
\sum\limits_{k=0}^{5}p_{k}(.5)\]
\[ =1 - \sum\limits_{k=0}^{5}{e^{-2}\frac{(2)^{n}}^{n!}} = \boxed{0.016}\]
\subsection*{(d)}
Since each customers arrival is uniformly distributed within the hour, both
customers have a $1/6$ independent probability of arriving in the last ten
minutes of the of the hour. Therefore the probability of them both arriving in
this time period is $\boxed{1/6^2}$
\subsection*{(e)}	
\[ 1/6 + 5/6*1/6 = \boxed{11/6^2}\]
\section*{Question 4}
\[ \rho = \frac{\lambda}{\mu}, \quad \rho = .8, \quad \mu = 2 \quad \Rightarrow
\lambda = 1.6\]
Although it is not explicit in the question I will make the assumption that the
2.5 minutes waiting for service is an \underline{average} queue time, that is : 
\[W_q = 2.5, \quad \boxed{R = 3} \]
Littles law state the average number is the system is 
\[L = \lambda W, \quad \boxed{L = 4.8}\]
\newpage
\section*{Question 5}
\[\lambda = 10, \quad \mu = 15, \quad \rho = \frac{\lambda}{\mu} = \frac{2}{3}
\]
\subsection*{(a)}
\[ p_0 = 1 - \rho = \boxed{1/3} \]
\subsection*{(b)}
\[ L_q = \frac{\rho^2}{1 - \rho} = \frac{\left(2/3\right)^2}{1 -
\left(2/3\right)} = \boxed{4/3}
\]
\subsection*{(c)}
\[E[R]= \frac{1}{\mu - \lambda} = \frac{1}{15 - 10} = \boxed{1/5  \quad
\mbox{*(12 minutes)}}\]

\subsection*{(d)}
\[0.125 = \frac{1}{\mu - 10} \Rightarrow 0.125\mu - 1.25 = 1 \Rightarrow
\boxed{\mu = 18 }\
\]

\subsection*{(e)}
\[(1-\rho) * 24  = \boxed{\mbox{ 8 papers an hour}}\]

\section*{Question 6}
\[\mu = 4, \quad W_q = .2 \mbox{ or 12 minutes}\]
\[.2 = \frac{\lambda}{4(4-\lambda)} \Rightarrowm \lambda = \boxed{16/9} \]

\section*{Question 7}
\[ \lambda = 3, \quad \mu = 7.5, \quad \rho = 0.4 \]

\subsection*{(a)}
\[\boxed{ \rho = 0.4 }\]
since if the service is being utilize they will have to wait and this is the
average utilization. 
\subsection*{(b)}
\[L_q = \frac{\rho^2}{1-\rho} = \frac{0.4^2}{1-0.4} = \boxed{4/15}\]
\subsection*{(c)}
\[\mbox{Prob}\{N > 5\} = 1 - \mbox{Prob}\{N \leq 5\}\]
\[= 1 -\sum\limits_{n=0}^{5}p_n = 1 - \sum\limits_{n=0}^{5}{\rho^n(1 -
\rho)}\]\[ = 1 - \sum\limits_{k=0}^{5}{0.4^k(1 -
0.4)} = \boxed{0.004096} \]
\subsection*{(d)}
\[W_q(t) = 1 - \rho e^{\mu (1- \rho)t} \]
\[\mbox{Prob}\{\mbox{Wait time} \leq \mbox{12 mins} | \mbox{the server is busy}
\} = W_q(0.20) = 1 - 0.4 e^{-7.5 (1- 0.4).20} = 0.8373 \] \[\mbox{Prob}\{
\mbox{Wait time} > \mbox{12 mins} \} = 0.6 + 0.4(0.8373) = \boxed{0.9349}\]
\subsection*{(e)}
\[8/60 = \frac{\lambda}{7.5(7.5 - \lambda)}\quad \Rightarrow \boxed{\lambda =
3.75}}\]

\section*{Question 8}
\section*{Question 9}
This is an example of an M/M/5 or M/M/6 queue respectively.
\[ \mbox{ 6/3/5 } : \lambda = 6, \quad \mu = 3, \quad c = 5, \quad \rho =
\frac{6}{5 * 2} = .6\]

\[ p_0 = 
  \left[ 1 + \left(\sum\limits_{n=1}^{c-1}{\frac{(c\rho)^n
  }{n!}}\right) + \frac{(c\rho)^c}{c!}\frac{1}{1 - \rho}\right]^{-1}\]

\[ p_0 = 
  \left[ 1 + \left(\sum\limits_{n=1}^{5-1}{\frac{(5 * .6)^n
  }{n!}}\right) + \frac{(5 * .6)^5}{5!}\frac{1}{1 - .6}\right]^{-1}\]
  
\[ \boxed{p_0 = 0.0466472} \]


\[ W_q = \left[
\frac{(\lambda/\mu)^c\mu}{(c - 1)!(c\mu - \lambda)^2} \right]p_0 \] 

\[W_q = \left[\frac{3(6/3)^5}{(5 - 1)!(5(3) - 6)^2}\right] * 0.0466472\]

\[ W_q = 0.0023032 \mbox{ hours } , \boxed{8.29 \mbox{ seconds}}\]
This should satisfy the requirement. 
\[ \mbox{ 6/3/6 } : \lambda = 6, \quad \mu = 3, \quad c = 6, \quad \rho =
\frac{6}{6 * 2} = .5\]

\[ p_0 = 
  \left[ 1 + \left(\sum\limits_{n=1}^{c-1}{\frac{(c\rho)^n
  }{n!}}\right) + \frac{(c\rho)^c}{c!}\frac{1}{1 - \rho}\right]^{-1}\]

\[ p_0 = 
  \left[ 1 + \left(\sum\limits_{n=1}^{5-1}{\frac{(5 * .5)^n
  }{n!}}\right) + \frac{(5 * .5)^5}{5!}\frac{1}{1 - .5}\right]^{-1}\]
  
\[ \boxed{p_0 = 0.0801001} \]


\[ W_q = \left[
\frac{(\lambda/\mu)^c\mu}{(c - 1)!(c\mu - \lambda)^2} \right]p_0 \] 

\[W_q = \left[\frac{3(6/3)^6}{(6 - 1)!(6(3) - 6)^2}\right] * 0.0801001\]

\[ W_q = 0.00089 \mbox{ hours } , \boxed{3.2 \mbox{ seconds}}\]
 \section*{Question 10}
\end{document}
