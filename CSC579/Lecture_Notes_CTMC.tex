%%This is a very basic article template.
%%There is just one section and two subsections.
\documentclass{article}
\usepackage{amsmath}
\usepackage{pgf}
\usepackage{lastpage}
\usepackage{amssymb}
\usepackage{tikz}
\usepackage[margin=0.75in]{geometry}
\usetikzlibrary{arrows,matrix,positioning}
\usepackage{listings}             % Include the  listings-package
\usepackage[utf8]{inputenc}
\usepackage[english]{babel}
\usepackage{fancyhdr}
\lstset{language=Matlab} 
\pagestyle{fancy}
\fancyhf{}
\fancyhead[LE,RO]{CTMC - Greg Timmons}
\fancyhead[RE,LO]{CSC 579 - Perf Modeling}
\fancypagestyle{plain}{\fancyfoot[LE,RO]{\thepage\backslash\pageref{LastPage}}} 
\fancyfoot[LE,RO]{\thepage\backslash\pageref{LastPage}}  
\usetikzlibrary{arrows,automata}
\usepackage[latin1]{inputenc}
\usepackage{pdfpages}
\begin{document}
	\title{Class Notes -- CTMC}
	\date{2/28/2015}
	\author{Gregory B Timmons, gbtimmon}
	\maketitle
\underline{CTMC} \\
 Continuous Time Markov chain, is a markov chain is a chain
where a state transtion can happen at infinitesimally small time periods -- at any
point in continuous time. We measure tranistion in the expected rate of
transitions.
\\
\\This markov chain is still \underline{memoryless}, in that neither previous
states or time spent in the current state relavent to the computation of
probability of transition to furture states
\\
\\Non-homogenous CTMC still depend on time in the chain. $p_{ij}(s,t)$ is the
probability having transitioned from state 'i' to state 'j' by time 't' when you
where in state i at time 's'.
\\
\\Similar to a DTMC, a homogenous chain does not depen on how long you have
been in the chain as transtion probabilities are no dependent on time. 

\end{document}